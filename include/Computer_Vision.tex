\chapter{Computer Vision}
\begin{center}
\vspace{-6ex}
\textit{"A picture is worth a thousand words"}
\vspace{6ex}
\end{center}
\label{chap:imgPro}

Before applying the data in the machine learning algorithms we need to clean the data we receive from the camera. The extent of the cleaning process depends on the learning algorithm and also the data itself.

\section{Images}

Using images is a way to display large amount of data visually. An image is often represented using a one or many 2-dimensional matrices. The building stone of an image is a pixel, which is a vector of size, $d$, where $d$ is the number of matrices used by the image or also called channels, $c$. The values of the pixels are usually called the intensity, $I_c$, of the corresponding channel and are limited to a certain set of values, which could be either continuous or discreet. The channels are usually represented using different colors and more often then not by red, green and blue colors, more on this can be read in Appendix.~\ref{chap:color_rep}.

\subsection{Information in images}

The number of channels used by an image is called the $depth$ of the image, and is as mentioned above, often 3. It is also mentioned that the intensities in these channels are limited to a set and most file formats, in which the images are stored, uses $\unit[8]{bits}$ to express these values, meaning each pixel contains $\unit[24]{bits}$.\footnote{Sometimes a fourth channel, the $alpha$ channel, is present in an image and thus a pixel contains $\unit[32]{bits}$ of information, but these extra bits do not carry any color information and will be omitted in this thesis.} This means that each pixel can contain any of $(2^8)^3=16777216$ different combinations of colors. The dimensions of the matrices of the image is usually of the order $~10^3$, meaning that there are a vast amount of different combinations of images. Surely there must be a way to classify the message of an image by considering different layers of information.

\begin{figure}[H]
    \centering
    \includegraphics[width=0.25\textwidth]{figure/computer_vision/images/first.png}
    \includegraphics[width=0.25\textwidth]{figure/computer_vision/images/overlap.png}
    \includegraphics[width=0.25\textwidth]{figure/computer_vision/images/second.png}
    \caption{\label{pic:overlap} Above we see two binary images of size $28 \times 28$ that both shows one representation of the digit 5. Individually there is no ambiguity of their illustrations, but studying their overlap in the middle picture they might as well come from different classes as the conjoined sections are far smaller than the secluded parts.}
\end{figure}

An image with a message does not carry pixels independently of each other, in Cref{pic:overlap} we can see two different images both containing enough information to convey a digit 5. So the images must contain more information than what is given by the individual pixels, the position of these must also play a role. Therefore we need to distinguish different parts of the image before we can be sure of its representation. These parts will be divided into three categories,

\begin{itemize}
    \item \textbf{Global}: All pixels must be considered simultaneously to carry a message.
    \item \textbf{Regional}: Only a subsection of the image is important for classification.
    \item \textbf{Local}: The individual pixel information is important.
\end{itemize}

and a combination of the regional and local parts will often be used. As we saw earlier in the example with the digits, the regional information of the white pixels were the important ones to see what the image showed and it's regional and not global, since a linear transformation of the white pixels would not change the characteristics.

\section{Image segmentation}

In order to extract useful information from an image, we need to segment the image into the parts we want to extract information for, e.g. if we want to study a plant in a field we need to remove the background dirt. So we need some way of extracting the local and regional information from the global.

\subsection{Histogram-based thresholding}

If the interesting part of the image is fundamentally different from the rest of the image, e.g. a white cloud on a blue sky, we can use a technique called thresholding. In thresholding, we represent the image using only one intensity, i.e. only one channel, and creates a threshold value, $T_I$. Using this threshold we create a binary image where 1 represents the distinct part of the image and 0 the indistinct part of the image using,

\begin{equation}
    I_S(x,y)=\left \{ \begin{array}{ll}
    1 & I(x,y) \geq T_I \\
    0 & I(x,y) < T_I
    \end{array} \right.
    \label{eq:thresholding}
\end{equation}

Choosing this threshold might be trivial for some images, e.g. images which have a bright and a dim region, but sometimes the intensities are very close or even overlapping. So in order to make the different part easily separable one would need to represent the image intensities using a non-trivial transformation from the different inputs. E.g. separating a green plant from a brown background could use the color information to separate these, but both parts might have some overlapping of the color channels. Instead we can transform the three channels, red, green, and blue to another index which separates the color information, this is done by converting the RGB color to HSI color space. Details of this transformation can be found in Appendix.~\ref{app:color}. The Hue index of this color space is the only index carrying information of which color it represents, which makes it an excellent choose for separating objects of different colors. To select the threshold we will then use a method called adaptive global threshold. The histogram of the intensities is first plotted in a graph, which should show two decently separated sections and a value between the two regions peaks should be selected as an initial estimate of the threshold.

Using Equation.~\eqref{eq:thresholding} we will get two different groups and the mean values of the intensities in respective group should be calculated. The threshold is then updated to be the mean of the means, meaning we should end up with with a threshold right between the two peaks. This will procedurally produce better and better estimates for the separating threshold, and the procedure will stop when it has converged or when the threshold updates less than a targeted change $\Delta T$. An example can be seen in Figure.~\ref{fig:adaptiveThreshold}.

\begin{figure}
    \centering
    % This file was created by matlab2tikz.
%
%The latest updates can be retrieved from
%  http://www.mathworks.com/matlabcentral/fileexchange/22022-matlab2tikz-matlab2tikz
%where you can also make suggestions and rate matlab2tikz.
%
\definecolor{mycolor1}{rgb}{0.00000,0.44700,0.74100}%
\definecolor{mycolor2}{rgb}{0.85000,0.32500,0.09800}%
\definecolor{mycolor3}{rgb}{0.92900,0.69400,0.12500}%
%
\begin{tikzpicture}

\begin{axis}[%
width=4.521in,
height=3.566in,
at={(0.758in,0.481in)},
scale only axis,
xmin=0,
xmax=1,
xlabel={Intensity, $I$},
ymin=0,
ymax=60000,
ylabel={Pixel intensity counts, \#},
axis background/.style={fill=white},
title style={font=\bfseries},
title={Example of adaptive global threshold}
]
\addplot [color=mycolor1,solid,forget plot]
  table[row sep=crcr]{%
0	55650\\
0.00392156862745098	0\\
0.00784313725490196	49060.2\\
0.0117647058823529	49130.8\\
0.0156862745098039	53664.2\\
0.0196078431372549	50399.8\\
0.0235294117647059	49678.4\\
0.0274509803921569	49305.2\\
0.0313725490196078	44722\\
0.0352941176470588	41254.2\\
0.0392156862745098	36860.2\\
0.0431372549019608	32789\\
0.0470588235294118	29840.2\\
0.0509803921568627	27157.2\\
0.0549019607843137	23784.2\\
0.0588235294117647	21756.4\\
0.0627450980392157	19380.2\\
0.0666666666666667	16204.4\\
0.0705882352941176	14961.2\\
0.0745098039215686	13291.2\\
0.0784313725490196	11498\\
0.0823529411764706	10476.8\\
0.0862745098039216	9980.4\\
0.0901960784313725	8628\\
0.0941176470588235	8053.6\\
0.0980392156862745	7121\\
0.101960784313725	6726.2\\
0.105882352941176	6159.8\\
0.109803921568627	5833.2\\
0.113725490196078	5447.4\\
0.117647058823529	5361.4\\
0.12156862745098	5133.2\\
0.125490196078431	4820.4\\
0.129411764705882	4758.4\\
0.133333333333333	4549.8\\
0.137254901960784	4437.4\\
0.141176470588235	4259.6\\
0.145098039215686	4209.4\\
0.149019607843137	3975.8\\
0.152941176470588	3998.8\\
0.156862745098039	3889.6\\
0.16078431372549	3958.4\\
0.164705882352941	3871\\
0.168627450980392	3808.4\\
0.172549019607843	3836.8\\
0.176470588235294	3879.2\\
0.180392156862745	3719.6\\
0.184313725490196	3693.8\\
0.188235294117647	3796\\
0.192156862745098	3791.4\\
0.196078431372549	3757\\
0.2	3851.4\\
0.203921568627451	3922.4\\
0.207843137254902	4013\\
0.211764705882353	3953.8\\
0.215686274509804	3955\\
0.219607843137255	3909.4\\
0.223529411764706	4131\\
0.227450980392157	4114.2\\
0.231372549019608	4231.6\\
0.235294117647059	4340.8\\
0.23921568627451	4406\\
0.243137254901961	4301.4\\
0.247058823529412	4663.6\\
0.250980392156863	4657.2\\
0.254901960784314	4642.4\\
0.258823529411765	5035\\
0.262745098039216	5135.2\\
0.266666666666667	4876.2\\
0.270588235294118	5207.8\\
0.274509803921569	5524.6\\
0.27843137254902	5553\\
0.282352941176471	5677.4\\
0.286274509803922	5944.8\\
0.290196078431373	5886.4\\
0.294117647058824	6226\\
0.298039215686275	6324.8\\
0.301960784313725	6645.2\\
0.305882352941176	6742.6\\
0.309803921568627	7003.6\\
0.313725490196078	6771\\
0.317647058823529	6982\\
0.32156862745098	7157.6\\
0.325490196078431	7109.8\\
0.329411764705882	7215.6\\
0.333333333333333	7604.8\\
0.337254901960784	7477.6\\
0.341176470588235	7304.2\\
0.345098039215686	7954.6\\
0.349019607843137	7907.2\\
0.352941176470588	7748.6\\
0.356862745098039	7679.6\\
0.36078431372549	7719.2\\
0.364705882352941	7443.8\\
0.368627450980392	7782.8\\
0.372549019607843	7440.2\\
0.376470588235294	7405.6\\
0.380392156862745	7652.6\\
0.384313725490196	7288\\
0.388235294117647	6708.4\\
0.392156862745098	6762\\
0.396078431372549	6841.6\\
0.4	6134.4\\
0.403921568627451	6121.4\\
0.407843137254902	5931.6\\
0.411764705882353	5603.2\\
0.415686274509804	5238.2\\
0.419607843137255	5243.4\\
0.423529411764706	4836.2\\
0.427450980392157	4702.2\\
0.431372549019608	4517\\
0.435294117647059	4241.4\\
0.43921568627451	3950\\
0.443137254901961	3772.4\\
0.447058823529412	3464\\
0.450980392156863	3326\\
0.454901960784314	3138.6\\
0.458823529411765	2862\\
0.462745098039216	2638.6\\
0.466666666666667	2521.4\\
0.470588235294118	2397\\
0.474509803921569	2266.8\\
0.47843137254902	2127.4\\
0.482352941176471	2036.8\\
0.486274509803922	1905.6\\
0.490196078431373	1726\\
0.494117647058824	1734.2\\
0.498039215686275	1686.6\\
0.501960784313725	1590.2\\
0.505882352941176	1505\\
0.509803921568627	1507.4\\
0.513725490196078	1324.4\\
0.517647058823529	1309.8\\
0.52156862745098	1280.4\\
0.525490196078431	1242.4\\
0.529411764705882	1184\\
0.533333333333333	1132.4\\
0.537254901960784	1074.4\\
0.541176470588235	1069.4\\
0.545098039215686	1071.2\\
0.549019607843137	1012.2\\
0.552941176470588	988.4\\
0.556862745098039	967.6\\
0.56078431372549	934\\
0.564705882352941	931.6\\
0.568627450980392	936\\
0.572549019607843	921.8\\
0.576470588235294	917.2\\
0.580392156862745	924\\
0.584313725490196	851.2\\
0.588235294117647	864.8\\
0.592156862745098	874.8\\
0.596078431372549	835.8\\
0.6	804\\
0.603921568627451	853.4\\
0.607843137254902	802\\
0.611764705882353	761.4\\
0.615686274509804	803.6\\
0.619607843137255	805\\
0.623529411764706	744.6\\
0.627450980392157	745\\
0.631372549019608	744.2\\
0.635294117647059	719.6\\
0.63921568627451	721.2\\
0.643137254901961	713.4\\
0.647058823529412	696\\
0.650980392156863	723\\
0.654901960784314	710.4\\
0.658823529411765	665.8\\
0.662745098039216	674.4\\
0.666666666666667	701.6\\
0.670588235294118	663.8\\
0.674509803921569	651.2\\
0.67843137254902	652\\
0.682352941176471	672.2\\
0.686274509803922	640\\
0.690196078431373	654\\
0.694117647058824	663.8\\
0.698039215686274	669\\
0.701960784313725	637.6\\
0.705882352941177	643.8\\
0.709803921568627	643.4\\
0.713725490196078	632.8\\
0.717647058823529	621.6\\
0.72156862745098	616.8\\
0.725490196078431	627.2\\
0.729411764705882	605.2\\
0.733333333333333	578.8\\
0.737254901960784	605.6\\
0.741176470588235	598.8\\
0.745098039215686	565\\
0.749019607843137	561.4\\
0.752941176470588	577.4\\
0.756862745098039	530.6\\
0.76078431372549	548.4\\
0.764705882352941	557.6\\
0.768627450980392	556.8\\
0.772549019607843	538.6\\
0.776470588235294	567\\
0.780392156862745	542.8\\
0.784313725490196	535\\
0.788235294117647	548.2\\
0.792156862745098	571.8\\
0.796078431372549	542.2\\
0.8	550.6\\
0.803921568627451	538.8\\
0.807843137254902	541.8\\
0.811764705882353	528.8\\
0.815686274509804	529.6\\
0.819607843137255	519.6\\
0.823529411764706	521.2\\
0.827450980392157	525.4\\
0.831372549019608	512\\
0.835294117647059	517.2\\
0.83921568627451	525\\
0.843137254901961	527.6\\
0.847058823529412	516.8\\
0.850980392156863	516.8\\
0.854901960784314	524.8\\
0.858823529411765	517\\
0.862745098039216	530.4\\
0.866666666666667	524.2\\
0.870588235294118	537.2\\
0.874509803921569	528.2\\
0.87843137254902	539.8\\
0.882352941176471	534.2\\
0.886274509803922	518.2\\
0.890196078431373	501.6\\
0.894117647058824	514.8\\
0.898039215686275	520.8\\
0.901960784313726	515.4\\
0.905882352941176	537.6\\
0.909803921568627	538.4\\
0.913725490196078	545.6\\
0.917647058823529	516.2\\
0.92156862745098	512.8\\
0.925490196078431	515\\
0.929411764705882	521.2\\
0.933333333333333	500.6\\
0.937254901960784	518.2\\
0.941176470588235	520\\
0.945098039215686	495.8\\
0.949019607843137	495.6\\
0.952941176470588	494\\
0.956862745098039	486.8\\
0.96078431372549	486.6\\
0.964705882352941	501.2\\
0.968627450980392	499.6\\
0.972549019607843	491.2\\
0.976470588235294	472.2\\
0.980392156862745	463\\
0.984313725490196	352\\
0.988235294117647	252.4\\
0.992156862745098	166.2\\
0.996078431372549	0\\
1	0\\
};
\addplot [color=mycolor2,solid,forget plot]
  table[row sep=crcr]{%
0.1	0\\
0.1	6726.2\\
};
\node[above right, align=left, text=black]
at (axis cs:0.093,6726.2) {$T_1$\\$\downarrow$};
\addplot [color=mycolor3,solid,forget plot]
  table[row sep=crcr]{%
0.195189794737765	0\\
0.195189794737765	3757\\
};
\node[above right, align=left, text=black]
at (axis cs:0.188,3757) {$T_2$\\$\downarrow$};
\end{axis}
\end{tikzpicture}%
    \caption{\label{fig:adaptiveThreshold}The initial threshold guess, $T_1$, is too far to the left for properly segment the two peaks. The threshold is adjusted to $T_2$ after one iteration of the adaptive global threshold algorithm.}
\end{figure}

\section{Image processing}

More often than not, the image thresholding is not enough to completely separate the object from the background. Often there is other objects that are of similar color that come as a side product of the thresholding and also parts of the object might not perfectly separated. In this section we will deal with the post processing of the image thresholding. Here we will use tools from a mathematical concept called morphological operations. These operations will also be useful in the next chapter where we will extract useful information of the object obtained after this post-process.

\subsection{Separated parts connection}

The first part of the post-processing will be to connect separated parts of the object that are close to each other, and the concept is to make all the objects found using the thresholding slightly larger, after which the parts of the desired should be connected. We apply the morphological operation, dilation, to the previously acquired binary image using a $3x3$ structure element,

\begin{equation}
    SE=\left[
        \begin{array}{ccc}
        1 & 1 & 1 \\
        1 & 1 & 1 \\
        1 & 1 & 1 \\
        \end{array}
    \right]
    \label{eq:se}
\end{equation}

\subsection{Connected components}

When we have connected the different parts of the object we want to extract the object by itself. We will do this by using the so called connected components, the general idea is to find different parts of the binary image which are connected with each other, and then the component which occupies the largest portion of the image should be our desired object. The idea is to search through the whole image, checking whether two neighbouring pixels belong to the foreground, and then set them as the same connected component. The binary image should then contain different clusters belonging to different groups.

\subsection{Hole filling}

We started this post-processing by connecting separated parts of the object, and then we extracted the object from other objects, but the result might still not represent the whole object as there might be some small sections in it that was not captured by the dilation process. We would like to fill these holes to get a completely solid object. This procedure is quite simple, we start by inverting the image, i.e. the white pixels are now black and vice versa. Then we extract the connected components. All the holes in the object should now be in different groups, and since we know that all the holes are inside the object, none of them are at the border of the image. So removing the connected components which has pixels at the borders will make sure only the holes are remaining, and to get the final result we add the images together.

\subsection{Object contraction}

We started this process by making the object slightly larger by the dilation operation, and to finalize it we will reverse this operation to remove the border that does not belong to the object. So as the last part of the post-processing part is to apply the opposite of the dilation, namely the erosion, using the same structure element from \eqref{eq:se}.
