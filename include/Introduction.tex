\chapter{Introduction}
\begin{center}
\vspace{-6ex}
\textit{"Boot up"}
\vspace{6ex}
\end{center}

Since the dawn of computers, humans have had access to computational power previously far beyond reach. In the beginning this was used to numerically solve Ordinary Differential Equations, (\textit{ODE}), that had no exact explicit solution. With mathematical computational, the computer was able to retrieve solutions to problems previously far beyond human reach. Although, with the immense computational power that the computer possessed it still lacked something that made man superior still. Intuition, creativity and being able to reason about results given prior knowledge for similar problems. As the computer had to be programmed and do exactly as programmed, it was as good as the code giving it instructions. This made the computer static, the program was never able to learn from previous problems.\\

This limitation is something that is progressively being erased today. Machine learning is a field of computer science, in which one studies algorithms that learn from data to make statistical predictions on new data. In this thesis, different machine learning algorithms are discussed and compared on an application in image recognition.

\section{Background}

Over-fertilization of crops and heavy use of herbicides in weed control introduces chemicals into the ecosystem. Reducing the amount of these substances is therefore of a problem requiring attention. In order to make best use of these products, local measurements of the required chemicals can be used for optimal distribution. For this to work, information about the system, as well as good analytic tools is needed to interpret the current state. Using cameras to acquire images over the fields will enable a database to be used in machine learning algorithms that will enable extraction of information over the system.

\section{Aim}

The primary goal of this thesis is to be able to extract and classify different kinds of weeds that are located in a farming field. Given an image of a section of a field, information about the field is in form of location and density of the weeds. This information will be crucial when determining the state of the field and what is the proper course of action in order to both maximize the yeild and growth of the intended plants.

\section{Limitations \textbf{senare}}

There exists different kinds of limitations within the framework of this project. The first limitations is due to small amount of data. In order to give machine learning algorithms a chance to do its work, a large amount of data is required to make proper assumptions on the dataset. For the initial stages of the project, a dataset of 8 different kinds of weeds are provided with 27 images each. On this dataset, the different algorithms are tested and evaluated.

\section{Temporär information}

\begin{table}[H]
\centering
\begin{tabular}{ll} \hline\hline
Name & Command\\ \hline
Chapter & \textbackslash\texttt{chapter\{\emph{Chapter name}\}}\\
Section & \textbackslash\texttt{section\{\emph{Section name}\}}\\
Subsection & \textbackslash\texttt{subsection\{\emph{Subsection name}\}}\\
Subsubsection & \textbackslash\texttt{subsubsection\{\emph{Subsubsection name}\}}\\
Paragraph & \textbackslash\texttt{paragraph\{\emph{Paragraph name}\}}\\
Subparagraph & \textbackslash\texttt{paragraph\{\emph{Subparagraph name}\}}\\ \hline\hline
\end{tabular}
\end{table}
