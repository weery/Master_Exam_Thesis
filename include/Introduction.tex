\chapter{Introduction}
\begin{center}
\vspace{-6ex}
\textit{"Boot up"}
\vspace{6ex}
\end{center}

Since the dawn of computers, humans have had access to computational power, magnitudes larger than before. In the beginning this was used to numerically solve Ordinary Differential Equations, (\textit{ODE}), that had no exact explicit solution. Through numerical computation, the computer was able to retrieve solutions to problems out of human computational range. Although, with the immense computational power that the computer possessed it still lacked something that made man superior still. Intuition, creativity and reason about results given prior knowledge for similar problems. As the computer had to be programmed and do exactly as programmed, it was as good as the code giving it instructions and the programmer behind them. This made the computer static, the program was not able to learn from previous problems.\\

This limitation is something that is progressively being erased today. Machine learning is a field of computer science, in which one studies algorithms that learn from data to make statistical predictions on new data. Modern algorithms even perform at a human level today, e.g. speech recognition programs performs almost as well as a human\cite{google}.  In this thesis, different machine learning algorithms are discussed and compared on an application in image recognition.

\section{Motivation}

Over-fertilization of crops and heavy use of herbicides in weed control introduces chemicals into the ecosystem. Reducing the amount of these substances is therefore a problem requiring attention. In order to make the best use of these products, local measurements of the required chemicals can be used for optimal distribution. For this to work, information about the system, as well as good analytical tools is needed to interpret the current state of the fields.
The start-up company FarmDrones had started to create a product that determines the state and health of using drones with a mounted camera. The drone would go over the entire field while taking images. Since images over the fields would be provided, a method that estimates the density and location of weeds are of interest. This could be used to know where to put out herbicides instead of using the same amount uniformly  across the whole field. Therefore they would want an algorithm that can both determine the kind of plant in an image and also where it is. Today this process is mostly done by humans. Although the farmers and their advisers know their trade and the fields, they do not have all information. This process usually progresses as follows, the farmers tells the adviser that they have found an area infested with weed, the adviser then comes to the field and makes a classification of the weed and an estimation of the infested area. The farmer receives a suggested action and then the weed removal can begin. This process contains many exchanges between the farmer and the adviser, and many estimations are made. If even one part of this could be automated, then it would be easier for the farmer to take the best course of action.

\section{Aim}

The previous section describes the intended goal of the project, use images from drones to determine where and how much weeds there are in a field. Since the start of the project, the use of drones with camera has been classified under the surveillance in Sweden. This means that using drones with a mounted camera is a tedious task as one needs to get approval from TransportStyrelsen each flight. This made so that FarmDrones was not able to continue its operation, which lead to that the initial motivation for the project has then been terminated. Instead images at eye level has been used for this project. Although the data for the project has been switched, the aim is still present, namely the creation of algorithms that can find and classify noxious weeds in farming fields. Similar work has been performed previously and one that will be closely followed is a another Master's thesis project by Mattias Andersson\cite{WeedClassification}. Mattias thesis was written in 1998, meaning there has been 19 years of new tools available and the first target of this thesis is to be able to reproduce his work using his methods, while also try and tackle the same problem using modern tools. Though, his work only targets the classification part of the project and for the weed location part has used another data set which was acquired early on in the project.

\section{Method}

The method that will be used to make the predictions on the data lies in the field of machine learning, from the method used by Mattias to the popular neural networks. Also, image processing methods will be used heavily in order to extract objects in images and to manipulate colors of the pixels. Two different programming languages will be used, the Matlab scripting language will be used as it is one of the best environments to develop and test algorithms and visualize data. The programming language C++ will also be used for the algorithms that require more computational power, which this language deals with better than Matlab since the program is compiled before running instead of the often slower interpreted program in Matlab. Another reason to use C++ its object oriented programming (OOP) nature, which makes it easier to create algorithms which generalize better.

\section{Limitations}

The first limitation for this project arose when it stopped using drones. At the same time, the intended use for the project was not purely for customer usage. The company conducting this project has since the start of this project been disbanded, so there is no reason why the result should be made as a customer focused product. This means that the thesis can focus more on the results rather than a product that is easy to use, although the results should be of main focus, this makes the process more complex. Another limitation is the amount of data available, the majority of the data acquired for this project has been gathered at one occasion and location meaning that the data is not very diverse.
